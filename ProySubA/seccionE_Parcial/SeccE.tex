\documentclass{article}
\usepackage{amsmath}

\usepackage{graphicx} % Required for inserting images
%\topmargin -.45in
\textwidth 6.5in
\textheight 9.in
\oddsidemargin 0in
\headheight 0in
\usepackage{float}
\usepackage{graphicx}
\usepackage{fancybox}
% \usepackage{p alatino}
\usepackage[utf8]{inputenc} %solucion del problema de los acentos.
\usepackage{epsfig,graphicx}
\usepackage{multicol,pst-plot}
\setlength{\columnsep}{5mm}
\usepackage{pstricks}
\usepackage{amsmath}
\usepackage{amsfonts}
\usepackage{amssymb}
\usepackage{eucal}
\usepackage[left=2cm,right=2cm,top=2cm,bottom=2cm]{geometry}
\pagestyle{empty}
\DeclareMathOperator{\tr}{Tr}
\newcommand*{\op}[1]{\check{\mathbf#1}}
\newcommand{\bra}[1]{\langle #1 |}
\newcommand{\ket}[1]{| #1 \rangle}
\newcommand{\braket}[2]{\langle #1 | #2 \rangle}
\newcommand{\mean}[1]{\langle #1 \rangle}
\newcommand{\opvec}[1]{\check{\vec #1}}
\renewcommand{\sp}[1]{$${\begin{split}#1\end{split}}$$}
\usepackage{hyperref}       % hyperlinks
\usepackage{url}            % simple URL typesetting
\usepackage{booktabs}       % professional-quality tables
\usepackage{amsfonts}       % blackboard math symbols
\usepackage{nicefrac}       % compact symbols for 1/2, etc.
\usepackage{microtype}      % microtypography
\usepackage{lipsum}
\usepackage[spanish, es-tabla]{babel}
\usepackage{amsmath}
\usepackage{amsfonts}
\usepackage{amssymb}
\usepackage{graphicx}
\usepackage{fancyhdr}
\usepackage[nottoc]{tocbibind}
\usepackage[titletoc]{appendix}
\usepackage{fancyhdr}
\usepackage{multirow}
\usepackage{subfigure}
\usepackage{csquotes}
\usepackage{csquotes}
\usepackage{multicol}
\usepackage{cancel}
\usepackage{mathtools}
\usepackage{amssymb}
\usepackage{parskip}
\usepackage{amsthm}
\usepackage{amsmath}
\usepackage{titlesec}
\usepackage{float}
\usepackage{multicol}
\graphicspath{{images/}}
\renewcommand{\labelitemii}{$\ast$}
\providecommand{\abs}[1]{\lvert#1\rvert}
\providecommand{\norm}[1]{\lVert#1\rVert}

\usepackage{listings}
\usepackage{color}

\definecolor{codegreen}{rgb}{0,0.6,0}
\definecolor{codegray}{rgb}{0.5,0.5,0.5}
\definecolor{codepurple}{rgb}{0.58,0,0.82}
\definecolor{backcolour}{rgb}{0.95,0.95,0.92}

\lstdefinestyle{mystyle}{
	backgroundcolor=\color{backcolour},   
	commentstyle=\color{codegreen},
	keywordstyle=\color{magenta},
	numberstyle=\tiny\color{codegray},
	stringstyle=\color{codepurple},
	basicstyle=\footnotesize,
	breakatwhitespace=false,         
	breaklines=true,                 
	captionpos=b,                    
	keepspaces=true,                 
	numbers=left,                    
	numbersep=5pt,                  
	showspaces=false,                
	showstringspaces=false,
	showtabs=false,                  
	tabsize=2
}


\begin{document}

\section*{Enunciado:}

Calcular el número de eventos de decaimiento beta por segundo para una muestra de carbono-14 ($^{14}$C) con un flujo incidente de $2 \times 10^{12}$ núcleos por segundo. La energía de los electrones de decaimiento es de 156 keV (kiloelectronvoltios). La sección eficaz para el decaimiento beta se considera $5 \times 10^{-5}$ barns. Suponiendo que la muestra contiene $6 \times 10^{23}$ átomos de carbono-14, determinar el número de eventos de decaimiento beta por segundo.

\section*{Solución:}

\[
\text{Número de eventos} = \text{Flujo incidente} \times \text{Sección eficaz} \times \text{Número de centros dispersores en la muestra}
\]

Primero, calculamos el número de centros dispersores en la muestra de carbono-14:

\[
\text{Número de centros dispersores en la muestra} = \frac{\text{Número de átomos en la muestra}}{\text{Número de Avogadro}}
\]
\[
\text{Número de centros dispersores en la muestra} = \frac{6 \times 10^{23} \text{ átomos}}{6.022 \times 10^{23} \text{ átomos/mol}} \approx 1 \text{ mol}
\]

Ahora, aplicamos la fórmula para calcular el número de eventos de decaimiento beta:

\[
\text{Número de eventos} = (2 \times 10^{12} \text{ núcleos/seg}) \times (5 \times 10^{-5} \text{ barns}) \times (1 \text{ mol}) \approx 10^{-4} \text{ eventos/seg}
\]

Este cálculo proporciona una estimación del número de eventos de decaimiento beta por segundo para la muestra de carbono-14 considerando la sección eficaz y el flujo incidente de núcleos por segundo.

\end{document}
